\documentclass{article}
\begin{document}
\subsection*{Thoughts}
    \begin{itemize}
        \item Remember: everytime you compile a pdf with latex, it creates a 
        change in the logfiles and you have to commit and push again although 
        it doesn't really seem like you made any changes
        \item All collaborators should have their own branch
        \item How do I change the environment used for python scripts using 
        the terminal? It does not work to change this in the Anaconda prompt 
        and open vs code from there.
        \item compare GitHub-for-education vs GitLab (UiB environment)
        \item implementation of GH+VSC:
        \item \begin{itemize}
            \item GEOF105 (Kjersti + Andrea+Julien) - autumn
            \item GEOF213 (Camille + Hari+?) - autumn
            \item GEOF346 (Helge + Kjersti?) - autumn
            \item Mostafa-course - autumn
            \item Thomas-course - autumn
            \end{itemize}
    \end{itemize}

    \subsection*{Existing repositories for UiB courses as of 02/2023:}
\begin{itemize}
    \item public: GEOF211, GEOF212, GEOF321, GEOF337
    \item private: GEOF105 (probably many more)
    \item maybe ask the owners of the repositories for potential more existing ones?
    \item professors (Ilker Fer, Mostafa Paskyabi, Kerim), researchers (Daniele, Ailin, Joao), PhDs, master students, etc
\end{itemize}

\subsection*{Brainstorm with Helge:}
\begin{itemize}
    \item show how to reach success with the tool for the teachers
    \item show how to use with MatLab with VSC
    \item INF100 course for students with Python over Anaconda
    \item have examples to go through, demonstrations (how to upload)
\end{itemize}

\pagebreak

\subsection*{Teachers' breakfast plan (35min)}:

\begin{itemize}
    % general presentation of the project (4min)
    \item Presentation of ourselves + General outline/aim of the talk (2min)
    \item Presentation of the issue (2min): Anaconda in INF100: only formal course for all MatNat students to learn Python
    % presentation of GH+VSC interface (16min)
    \item Presentation of the solution: GH+VSC (2min)
    \item Presentation of GH (2min)
    \item Presentation of VSC (2min)
    \item Demonstration of data synchronisation, push/pull/commit, etc (4min)
    \item Who uses this: presentation of existing repositories (1min)
    \item Quotes from GFI users in teaching (2min)
    \item Demonstration of VSC with MatLab (1min)
    \item Demonstration of VSC with Python (1min)
    \item Demonstration of VSC with Jupyter notebook (1min)
    % presentation of Vår notebook (10min)
    \item Presentation of Vår notebook (10min)
    % conclusion (5min)
    \item Recent research outputs/findings for GH and/or VSC and/or Jupyter notebooks (2min)
    \item Next steps with Tutorial guide-/handbook and/or course/seminar for teachers and students (1min)
    \item Next step with prototyping at GFI and maybe exporting outside to other departments at MatNat (1min)
    \item Thanks and segue into questions (1min)
\end{itemize}

\pagebreak

\subsection*{Cambridge University Press meeting:}

Mail from Thomas:\\
\indent
``It would probably be most valuable if you could talk about \textbf{why integrating programming would be valuable and how you intend to support it at our department}. In general, all future textbooks should have Python exercises that go along with the material. If this is a message we could land with her and with her understanding the need and potential, this could be useful.
If you can then speak both from your perspective as a student as well as from your perspective as someone developing the Python programming that goes along with our teaching."

\begin{itemize}
    \item GEOF210 (data analysis)/GEOF211 (numerical modeling) courses last moments to have learned programming for the students
    \item programming exercises printed in textbooks doesn't seem to make sense
    \item CUP could think about supporting textbook narrative/theory with relevant platform for programming environment (jupyter notebook, needs to be updated); if on GitHub, open source and peer-contributed but also free, could be good advertisement; if not, link file/folder to textbook for users. What's the best interface?
    \item current GEOF courses exercises could be relevant resources for a GFI-topic-textbook. We have several references at GFI.
\end{itemize}


\end{document}