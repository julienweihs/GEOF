\documentclass{article}

\title{An all-in-one framework to support student Python fluency at GFI with Jupyter notebooks}
\date{March 22nd, 2023}
\author{Vår Dundas \& Julien-Pooya Weihs}
\begin{document}

\maketitle

Despite the central role of programming in geophysics and the presence of Python activities throughout the bachelor curriculum at GFI, many students seem to struggle with this aspect of their studies. These difficulties can occur for different reason, and we propose to address at least two of them: the programming interface, and the first experiences with Python for the users. 

At this teachers breakfast session, we will introduce a framework allowing to write, interpret and compile various programming languages, to safely store the files on a cloud, and to facilitate easy collaboration via version control and task management. We will then demonstrate how this framework could be beneficial to students at GFI by presenting a set of coding exercises aimed at introducing the Python language through a series of step-by-step tutorials tailored for beginners.

The first part of the session will present the GitHub+Visual Studio Code environment, and go over its strengths and advantages. The second part will go over a selection of Jupyter notebooks used with large success in the Ekte Data project training science teachers. The presentation will be accompanied by real-time demonstrations and hands-on possibilities for the participants.

\end{document}