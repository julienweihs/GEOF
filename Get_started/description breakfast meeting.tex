\documentclass{article}

\title{
    An all-in-one framework to support student Python fluency at GFI with 
    Jupyter notebooks
    }
\date{March 22nd, 2023}
\author{Vår Dundas \& Julien-Pooya Weihs}
\begin{document}

\maketitle

Despite the central role of programming in geophysics and the presence of 
Python activities throughout the bachelor curriculum at GFI, many students 
struggle with this aspect of their studies. These difficulties can 
occur for different reason, and we propose to address at least two of them: 
the programming interface, and the user's first experiences with Python. 

At this teachers breakfast session, we will introduce a framework allowing to 
write, interpret, and compile various programming languages, to safely store 
the files on a cloud, and to facilitate easy collaboration via version control 
and task management. We will then demonstrate how this framework could be 
useful for students at GFI by presenting a set of coding exercises aimed at 
introducing the Python language to beginners.

During the first part of the session we'll present the GitHub+Visual Studio 
Code environment, and discuss its strengths and advantages. In the second part 
we'll go through a selection of Jupyter Notebooks used in a seminar series 
training high school teachers in Python through a Ekte Data/Skolelaboratoriet 
collaboration. The presentation will be accompanied by real-time 
demonstrations and hands-on possibilities for the participants.

\end{document}